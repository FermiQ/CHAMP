%\documentclass[a4paper,11pt]{article}
\documentclass[letter,11pt]{article}

%Fonts
\usepackage[english]{babel}
\usepackage[utf8]{inputenc}      %for accentuated letters
\usepackage[T1]{fontenc}         %important for correct cesure of accentuated letter
\usepackage{ae,aecompl}          %important for fonts in pdf

%Graphics
\usepackage{graphicx}  % Include figure files
\usepackage{dcolumn}   % Align table columns on decimal point
\usepackage{rotating}
\usepackage{epstopdf}  %automatic ps -> pdf conversion
\usepackage{tikz}
\usepackage{tikz-qtree}

%Maths
\usepackage{amstext,amsmath,amssymb,amsfonts}
\usepackage{bm}
\usepackage{physics}
\usepackage{enumitem}

%References
\usepackage{cite}     %to group multipled references
\usepackage{hyperref}

%Headers and Margins
\usepackage{fancyhdr}
\setlength{\topmargin}{-1cm}     %top margin
\setlength{\textheight}{24cm}    %bottom margin
\setlength{\textwidth}{16cm}     %right margin
\setlength{\oddsidemargin}{0cm}  %left margin (for odd pages or single sided)
\setlength{\evensidemargin}{0cm} %left margin (for even pages if using twoside)
%Margins
\setlength{\topmargin}{-3cm}      %top margin
\setlength{\textheight}{25.5cm}   %bottom margin
\setlength{\textwidth}{18cm}      %right margin
\setlength{\oddsidemargin}{-1cm}  %left margin (for odd pages or single sided)
\setlength{\evensidemargin}{-1cm} %left margin (for even pages if using twoside)

%Glossary entries
\usepackage{xstring}
\newcommand\entry[2]{
  % transform #1=``[comment] actual_name (list)''
  % into   \temp=``actual_name'' only
  \saveexpandmode\noexpandarg
  \vspace*{.5em}
  \IfBeginWith{#1}{[}{
    \IfSubStr{#1}{(}{
      \StrBetween{#1}{] }{(}[\temp]
    }{
      \StrBehind{#1}{] }[\temp]
    }
  }{
    \IfSubStr{#1}{(}{
      \StrBefore{#1}{(}[\temp]
    }{
      \StrSubstitute{#1}{\_}{_}[\temp]
    }
  }
  % title of the entry
  \noindent\underline{\textbf{\temp}} (usage: {\tt #1})
  % label and toc entry
  \label{entry:\temp}
  \phantomsection
  \addcontentsline{toc}{paragraph}{\temp}
  \vspace*{.5em}

  % content of the entry
  \noindent\begin{tabular}{m{.001\linewidth}m{.999\linewidth}}
    & \begin{minipage}{\linewidth}#2\end{minipage}
  \end{tabular}
  \restoreexpandmode
}

% Link to glossary entry
\newcommand\link[1]{
  \hyperref[entry:#1]{{\tt #1}}
}
\usepackage{underscore}
% Definition style
\newcommand\Def[1]{\underline{#1}}
% Object description
\renewcommand\line[2]{
  \noindent>\underline{\textbf{#1}}

  \noindent\begin{tabular}{p{.001\linewidth}p{.999\linewidth}}
    & \begin{minipage}{\linewidth}#2\end{minipage}
  \end{tabular}
  \vspace*{.2em}
}





\begin{document}

\title{Notes on the programming style in CHAMP}

\author{Julien Toulouse\\
Laboratoire de Chimie Th\'eorique\\Universit\'e Pierre et Marie Curie et CNRS, 75005 Paris, France\\
julien.toulouse@upmc.fr}

\date{\today}

\maketitle

\setcounter{secnumdepth}{5}
\setcounter{tocdepth}{5}
%\tableofcontents
\vspace*{5em}

These notes describe the programming paradigm used in the Fortran 90 files of the program CHAMP. This programming paradigm was invented by the theoretical chemist and legendary programmer François Colonna and was given different names over the years:
Open Structured Interfaceable Programming Environment (OSIPE)~\cite{ColJolPoiAngJan-CPC-94},
Deductive Object Programming~\cite{Col-ARX-06},
Implicit Reference to Parameters (IRP)~\cite{Sce-ARX-09},
as well as Minimal Explicit Declaration (MED).
Even though we describe a Fortran implementation of it, this programming paradigm is not restricted to any particular programming language, and in fact has been implemented in many languages. The particular Fortran implementation that I made in CHAMP was very much inspired by the implementation in the program QMCMOL~\cite{Qmc-PROG-XX} mainly made by Roland Assaraf.

\section{Introduction}
A Fortran computer program produces \Def{objects}, which are Fortran variables, of any type or dimension (scalar/array), containing quantities that we are after. For example, {\tt psi} may be an object containing the value of a wave function evaluated at some electron coordinates. This object {\tt psi} is constructed from other objects. For example, for the case of a Jastrow $\times$ single determinant wave function, we have {\tt psi = jastrow * determinant} where {\tt jastrow} and {\tt determinant} are the values of the Jastrow factor and of the Slater determinant at the same electron coordinates. The later objects are themselves constructed from yet other objects. For example, {\tt determinant} may be constructed from {\tt orbitals} which is an array containing the values of the orbitals at the considered electron coordinates. And so on.

Clearly, there are \Def{dependencies} between these objects, in the sense that the construction of a given object requires that other objects have already been constructed. In other words, the order in which the objects are constructed is important. Usually, in Fortran programs, these dependencies are {\it not} made explicit, and making sure that the order of the construction of the objects is correct is left to the programmer. In large codes, this may make difficult the implementation of a new object.

In the programming paradigm used here, the dependencies between the objects are made explicit. The programmer then does not need to take care of the order of construction of the objects. This facilitates the implementation of new objects.

\section{Basis of the paradigm and implementation}

The programming paradigm is best explained on a simple example. 

\subsection{Dependency tree and building subroutines}

Consider again the object {\tt psi}, constructed from two objects {\tt jastrow} and {\tt determinant}, themselves constructed from other objects according to the following \Def{dependency tree}:


\begin{center}
\begin{tikzpicture}
\Tree [.\node[draw]{\tt psi};
  [.\node[draw]{\tt jastrow};     
    [.{\tt jastrow_parameters} ]
    [ [ [ .\node(coord){\tt electron_coordinates}; ] ] ] ]
  [.\node[draw]{\tt determinant}; 
    [.\node[draw]{\tt orbitals};
      [.\node[draw](basis){\tt basis_functions}; [ {\tt basis_parameters} ] ] [.{\tt coefficients}
      ] ] ] ]
\draw (coord)--(basis);
\end{tikzpicture}
\end{center}

Note that, strictly speaking, this is not a tree but a graph since the branches are not necessarily disjoint. For example, in the example above, the object {\tt electron_coordinates} connects the two branches. We will nevertheless use the vocabulary of computer trees.

This tree encodes the dependencies between the objects.
For example, {\tt psi} depends on both {\tt jastrow} and {\tt determinant}.
In turn, {\tt jastrow} depends on {\tt jastrow_parameters} and {\tt electron_coordinates}. And so on.
In general, the elements of a tree are called \Def{nodes}.
We will say that {\tt psi} is a \Def{child} of {\tt jastrow} and {\tt determinant},
and {\tt jastrow} and {\tt determinant} are the \Def{parents} of {\tt psi}.
The objects without any parents (those that are not boxed: {\tt jastrow_parameters}, {\tt electron_coordinates}, {\tt basis_parameters}, {\tt coefficients}) are the \Def{leaves} of the tree.
These generally correspond to input parameters which must be given by the user when running a calculation.

\vspace{1em}
The practical implementation of the dependency tree in the program is explained in the next section. For the moment, let us describe how calculations are done, assuming that the information about the dependency tree is available in the program.
Each object which is not a leaf of the tree has a \Def{building subroutine} which constructs it. So, written schematically, we have the five following building subroutines in the program, corresponding to the five boxed objects of the tree:

\vspace{0.5cm}
\noindent
{\tt subroutine psi_bld\\
psi = jastrow * determinant\\
end subroutine psi_bld}

\vspace{0.5cm}
\noindent
{\tt subroutine jastrow_bld\\
jastrow  = $f(${\tt jastrow_parameters}$,${\tt electron_coordinates}$)$\\
end subroutine jastrow_bld}

\vspace{0.5cm}
\noindent
{\tt subroutine determinant_bld\\
determinant  = $f(${\tt orbitals}$)$\\
end subroutine determinant_bld}

\vspace{0.5cm}
\noindent
{\tt subroutine orbitals_bld\\
orbitals  = $f(${\tt basis_functions}$,${\tt coefficients}$)$\\
end subroutine orbitals_bld}

\vspace{0.5cm}
\noindent
{\tt subroutine basis_functions_bld\\
basis_functions  = $f(${\tt electron_coordinates}$,${\tt basis_parameters}$)$\\
end subroutine basis_functions_bld}

\vspace{0.5cm}
\noindent
Here, to simplify, we have shown the explicit expression of {\tt psi} only, the expressions of the other objects are just written as (complicated) functions $f$.

\vspace{0.5cm}
The most part of the program is made of building subroutines, but they do {\it not} make {\it all} the program. Here are examples of subroutines which are {\it not} building subroutines:
\begin{itemize}
\item The subroutines reading the input. This is usually where the leaf objects are created.
\item The subroutines writing the output.
\item The subroutines containing the main iterative algorithms, such as the Monte Carlo algorithm or the wave-function optimization algorithm.
\end{itemize}

\subsection{Implementation of the dependency tree (object_create and object_needed)}

In CHAMP, it was chosen to give the data of the dependencies between the objects directly in the building subroutines. The advantage of this is that all the information about a given object is localized in its building subroutine, and not delocalized in different places in the code.
In practice, each building subroutine contains a \Def{header} where the local information of the dependency tree is explicitly given.
The header of all the building subroutines is executed only once at the beginning of the execution of the program (see section \ref{sub:begin}) in order to construct the dependency tree. Then, in the actual calculations, we always have {\tt header=.false.} so the header is bypassed.

The object \Def{created} by a building subroutine is indicated by a call to {\tt object_create} and all the parent objects \Def{needed} by the routine are listed by calls to {\tt object_needed}.
Thus, the building subroutines of our simple example actually look like:

\vspace{0.5cm}
\noindent
{\tt subroutine psi_bld\\
if(header) then\\
\phantom{xx} call object_create(`psi')\\
\phantom{xx} call object_needed(`jastrow')\\
\phantom{xx} call object_needed(`determinant')\\
\phantom{xx} return\\
endif\\
psi = jastrow * determinant\\
end subroutine psi_bld}

\vspace{0.5cm}
\noindent
{\tt subroutine jastrow_bld\\
if(header) then\\
\phantom{xx} call object_create(`jastrow')\\
\phantom{xx} call object_needed(`jastrow_parameters')\\
\phantom{xx} call object_needed(`electron_coordinates')\\
\phantom{xx} return\\
endif\\
jastrow  = $f(${\tt jastrow_parameters}$,${\tt electron_coordinates}$)$\\
end subroutine jastrow_bld}

\vspace{0.5cm}
\noindent
{\tt subroutine determinant_bld\\
if(header) then\\
\phantom{xx} call object_create(`determinant')\\
\phantom{xx} call object_needed(`orbitals')\\
\phantom{xx} return\\
endif\\
determinant  = $f(${\tt orbitals}$)$\\
end subroutine determinant_bld}

\vspace{0.5cm}
\noindent
{\tt subroutine orbitals_bld\\
if(header) then\\
\phantom{xx} call object_create(`orbitals')\\
\phantom{xx} call object_needed(`basis_functions')\\
\phantom{xx} call object_needed(`coefficients')\\
\phantom{xx} return\\
endif\\
orbitals  = $f(${\tt basis_functions}$,${\tt coefficients}$)$\\
end subroutine orbitals_bld}

\vspace{0.5cm}
\noindent
{\tt subroutine basis_functions_bld\\
if(header) then\\
\phantom{xx} call object_create(`basis_functions')\\
\phantom{xx} call object_needed(`electron_coordinates')\\
\phantom{xx} call object_needed(`basis_parameters')\\
\phantom{xx} return\\
endif\\
basis_functions  = $f(${\tt electron_coordinates}$,${\tt basis_parameters}$)$\\
end subroutine basis_functions_bld}

\vspace{1em}
Often, it is convenient to construct in the same building subroutine several objects. A given building subroutine thus generally creates several objects. For example, if the building subroutine {\tt jastrow_bld} creates the two object {\tt jastrow} and {\tt jastrow_gradient}, it will look like:

\vspace{0.5cm}
\noindent
{\tt subroutine jastrow_bld\\
if(header) then\\
\phantom{xx} call object_create(`jastrow')\\
\phantom{xx} call object_create(`jastrow_gradient')\\
\phantom{xx} call object_needed(`jastrow_parameters')\\
\phantom{xx} call object_needed(`electron_coordinates')\\
\phantom{xx} return\\
endif\\
jastrow  = $f(${\tt jastrow_parameters}$,${\tt electron_coordinates}$)$\\
jastrow_gradient  = $g(${\tt jastrow_parameters}$,${\tt electron_coordinates}$)$\\
end subroutine jastrow_bld}

\vspace{0.5cm}
With this extension, the nodes of the dependency tree are no longer the objects themselves but instead the building subroutines, each one creating a list of objects. This extension does not cause any problem.

\subsection{Manipulating objects (object_provide and object_modified)}

All subroutines can manipulate the objects of the dependency tree with calls to the two main commands explained here: {\tt object_provide} and {\tt object_modified}.

%\subsection{call object\_provide}
\vspace{1em}

For example, suppose that the programmer wants to print out the value of {\tt psi} at some place in the code (outside the building subroutines). He just needs to write the following lines:

\vspace{0.5cm}
\noindent
{\tt call object_provide(`psi')\\
write(6,*) `psi=',psi}

\vspace{0.5cm}
The subroutine {\tt object_provide} checks if an object is \Def{valid}.
A valid object is an object that has already been calculated and can be used,
that is to say that it was calculated from objects that themselves are still valid
(see below {\tt object_modified} to invalidate an object).

Hence, {\tt object_provide(`psi')} behaves has follow:
\begin{itemize}
\item if {\tt psi} is valid, nothing is done. 
\item otherwise, the program checks if its parents {\tt jastrow} and {\tt determinant} are valid. 
\begin{itemize}
\item if both are valid, the program calls the building subroutine {\tt psi_bld} and marks {\tt psi} as valid too.
\item if, for instance, {\tt determinant} is not valid, the program checks if its parent {\tt orbitals} is valid.
      If {\tt orbitals} is valid, the program calls the building subroutine {\tt determinant_bld},
      marks {\tt determinant} as valid,
      then calls the building subroutine {\tt psi_bld} and marks {\tt psi} as valid.
      If {\tt orbitals} is not valid, the programs checks its parents, and so on.
\end{itemize}
\end{itemize}

In other words, {\tt object_provide(`psi')} goes down recursively the dependency tree under {\tt psi} until it finds valid objects. It then climbs up the dependency tree, constructing the objects one after the other, in the correct order, until it finally constructs {\tt psi}.

In the case where {\tt object_provide(`psi')} goes down to a leaf object of the dependency tree (for example, {\tt basis_parameters}) which is {\it not} valid, then it gives an error message indicating that this leaf object is necessary to construct {\tt psi} but does not know how to construct this leaf object. Indeed, since a leaf object does not have a building subroutine, the program does not know how to construct it. Leaf objects are instead usually read in from the input file at the beginning of execution and marked as valid then.

In summary, {\tt object_provide(`psi')} has several advantages:
\begin{itemize}
\item It is {\it simple}. The programmer just needs to know the name of the object that he wants, here `psi'. He does not need to know how this object is calculated by the program. He does not need to know about intermediate objects such as {\tt orbitals}.
\item It is {\it safe}. If a necessary leaf object is not available, the program will properly stops and explains what is missing.
\item It is {\it efficient}, in the sense that this mechanism ensures that only what is needed is calculated, nothing more.
\end{itemize}

%\subsection{call object\_modified}
\vspace{0.5cm}

Another important ingredient remains to be explained, and that is how to invalidate an object, i.e.
what if the value of an object, say {\tt electron_coordinates}, is modified?
We need to make sure that if we need any child or grandchild object of {\tt electron_coordinates},
this object must be recalculated with the new value of {\tt electron_coordinates}.
This is done by inserting, just after modifying {\tt electron_coordinates},
a call to the routine {\tt object_modified}:

\vspace{0.5cm}
\noindent
{\tt electron_coordinates = (-2.1, 0.7, 1.5)\\
call object_modified(`electron_coordinates')
}

\vspace{0.5cm}
The subroutine {\tt object_modified(`electron_coordinates')} marks {\tt electron_coordinates} as valid, and recursively climbs {\it up} the dependency tree to mark as \Def{invalid} all the objects depending on {\tt electron_coordinates}, namely {\tt jastrow},  {\tt basis_functions}, {\tt orbitals}, {\tt determinant}, {\tt psi}. 
Thus, if any one of these objects is asked for afterwards, it will be recalculated with the new value of {\tt electron_coordinates}. This is a {\it safe} mechanism since it prevents the programmer from forgetting to update objects in an iterative algorithm, a frequent bug otherwise.

Note that a typical use of {\tt object_modified} is to mark leaf objects as valid after obtaining them.
As stated before, leaf objects can be objects read from the input, but they can be objects coming from the non-MED part of the code (see below).

\section{Interface with the non-MED part}

The objects created in the non-MED files do not have building subroutines. If they are not available, the program does not know how to construct them. Nevertheless, to facilitate their use in the MED part, they can be added as leaf objects of the dependency tree. For this, we just need to add a call to {\tt object_modified}, for example in the line just after a non-MED object has been constructed. This provides a safe mechanism for using them, making sure that we use them only when they have been already calculated. In fact, I found this extremely useful when I started to program in CHAMP. 
When this is judged useful, a non-MED object can be converted to a MED object, i.e. we write a proper building subroutine for it and it is then a normal object of the dependency tree. The non-MED part of the code can then be progressively evolved in the style of the MED part.


\section{More detail}

\subsection{Dynamic allocations inside the building subroutines}

If an object is an array, it is dynamically allocated inside its building subroutine, using the routine {\tt object_alloc}. For example:

\vspace{0.5cm}
\noindent
{\tt call object_alloc(`orbitals', orbitals, norb)\\
do i=1,norb\\
\phantom{xx} orbitals(i) = ...\\
enddo}
\vspace{0.5cm}

Note that the dimension {\tt norb} itself is usually an object handled by the dependency tree. If it changes during the calculation, the array will be reallocated with the new size.

\subsection{Indexes of objects}
The routines {\tt object_modified} and {\tt object_provide} take as argument the name of an object given as a string. The name of the object is then looked up in an array which can take some time. If these functions are called in a part of the code that is very often executed, this is not efficient. In this case, we use directly the index of the object in the array, instead of its name.
Hence, a call to {\tt object_modified(`index')} can be substituted for a call to {\tt call object_modified_by_index(`index')}, and a call to {\tt object_provide(`name')} for a call to {\tt object_provide_by_index(`index')}


\section{Programming Manual}

\subsection{What happens at the beginning of a run?}
\label{sub:begin}

In this section, we describe how the MED implementation gathers the MED-related information (tree, address of the routines, etc\dots)

At the very beginning of a run, in the file {\tt champ.f90}, the program successively calls the routines {\tt build_tree}, {\tt catalog_objects} and {\tt define_averages_and_errors}.

The routine {\tt build_tree} calls for {\tt catalog_all_nodes_and_routines}, which is a manually made list of calls to {\tt catalog_one_node}. This last routine maps a routine string name with the routine address in the {\tt nodes} array (see section~\ref{subsub:types}). Then, {\tt build_tree} calls for {\tt execute_node_headers}, which calls for the execution of the headers of all the nodes now known in the {\tt nodes} array.

The routine {\tt catalog_objects} is a manually made list of calls to {\tt catalog_one_object}. This last routine maps an object string name to an object index in the {\tt objects} array (see section~\ref{subsub:types}). This actually needs only to be done on objects that are often used and for which the explicit knowledge of the index as a Fortran variable is useful (this explicit knowledge allows to bypass the search through objects for the index everytime a MED command is called).

The routine {\tt define_averages_and_errors} is a manually made list of calls to routines {\tt object_average_define} and {\tt object_error_define}.
HERE

\subsection{Glossary of the MED commands}

In Fortran, we cannot store the name of an object in a variable : once the code is compiled, the name is lost.
If we want to operate on dependencies, we need to retain the names, i.e. we need to have a mapping between each name and a variable.
In the CHAMP's implementation of the MED paradigm, there are Fortran objects, and they in general have a string {\tt `name'} and an integer {\tt `index'}.

Note that almost all basic MED instructions are given with the {\tt `name'} variable.
When a {\tt `name'} is given, the MED machinery searches for the corresponding {\tt `index'} ; in the core machinery, everything is done with the indexes.
Since the search can be costly, as stated before, for repetively used Fortran objects, it is possible to directly give the {\tt `index'}.

In the following sections, we give a glossary of basic and more advanced MED commands implemented in CHAMP.

\subsubsection{Types}
\label{subsub:types}

To well understand the commands described in the next sections, one might first want to understand how information about the objects and nodes are stored.

\entry{[fortran type] `object'}{
  The information on an object are stored into a type {\tt object}.
  All the objects are accumulated in an array {\tt objects} of type {\tt object},
  and the object index is the position of an object in that array.
  The type {\tt object} contains:

  \vspace{.5em}
  \noindent\begin{tabular}{m{.26\linewidth}m{2mm}m{.37\linewidth}m{.33\linewidth}}
    {\tt character            }&::& {\tt name                           }& its name\\
    {\tt logical              }&::& {\tt associated                     }& is it associated?\\
    {\tt character            }&::& {\tt type                           }& its type\\
    {\tt integer, allocatable }&::& {\tt dimensions (:)                 }& its dimensions\\
    {\tt logical              }&::& {\tt walkers = .false.              }& \\
    {\tt logical              }&::& {\tt unweighted = .false.           }& \\
    {\tt logical              }&::& {\tt valid                          }& is it valid?\\
    {\tt logical              }&::& {\tt freezed = .false.              }& is it freezed?\\
    {\tt logical              }&::& {\tt object_depend_valid = .false.  }& \\
    {\tt integer              }&::& {\tt node_create_index = 0          }& its building routine\\
    {\tt integer, allocatable }&::& {\tt nodes_depend_index (:)         }& \\
    {\tt pointer              }&::& {\tt pointer_[type]                 }& \\
    {\tt logical              }&::& {\tt saved    = .false.             }& \\
    {\tt real                 }&::& {\tt save_[type]                    }& \\
    (and statistical stuff)
\end{tabular}
}

\entry{[fortran type] `node'}{
  The information on a node (a building routine) are stored into a type {\tt node}.
  All the nodes are accumulated in an array {\tt nodes} of type {\tt node}.
  The type {\tt node} contains:

  \vspace{.5em}
  \begin{tabular}{m{.26\linewidth}m{2mm}m{.37\linewidth}m{.33\linewidth}}
    {\tt character            }&::& {\tt routine_name                  }& its name \\
    {\tt integer              }&::& {\tt routine_address               }& its address \\
    {\tt integer, allocatable }&::& {\tt objects_needed_index(:)       }& the list of needed objects \\
    {\tt integer, allocatable }&::& {\tt objects_create_index(:)       }& the list of created objects \\
    {\tt logical              }&::& {\tt valid                         }& is it valid? \\
    {\tt logical              }&::& {\tt debug = .false.               }&  \\
    {\tt logical              }&::& {\tt entered = .false.             }&  \\
    {\tt integer              }&::& {\tt calls_nb = 0                  }& stat data \\
    {\tt real                 }&::& {\tt cpu_duration = 0.d0           }& stat data \\
  \end{tabular}
}

\vspace{1em}
\noindent Hence, for example:

    {\tt nodes(objects(object_index)\%node_create_index)\%routine_name}

\noindent will give you the {\tt routine_name} of
the routine whose index is stored in {\tt node_create_index} of
the current object (known by its {\tt object_index}):
this is the name of the routine creating the object of index {\tt object_index} !


\subsubsection{Basics}

In this section, we describe in a glossary manner the basic commands of the MED implementation in CHAMP.
Those are the commands that are to be called when writting new code or modifying code.
In particular they can be found in all the files in CHAMP,
by contrast to the ``deeper machinery'' commands described in the next section,
that are (almost) only called in the ``{\tt MED_tools/*f90}'' files themselves.

\vspace{1em}

\entry{[sub] object_create(`name'[,\tt `index'])}{
  Catalogs an object known by its {\tt `name'}
  with the current building node index, 
  and possibly returns its (newly produced) {\tt `index'}.

  This uses \link{object_add_once_and_index} to obtain the `index' of the object.
  \vspace{1em}

  In the code:

  \begin{itemize}
  \item register that the current node creates the object
  \\    ({\tt objects(index)\%node_create_index = node_current_index})
  \item and that the object is created by the current node
  \\    (add {\tt index} to {\tt nodes(node_current_index)\%objects_create_index})
  \end{itemize}
}

\entry{[sub] object_needed(`name')}{
  Catalogs a dependency of the current building node.

  This uses \link{object_add_once_and_index} (in case the object itself is not already catalogued).
  \vspace{1em}

  In the code:

  \begin{itemize}
  \item register that the object is needed by the current node
  \\    (add {\tt object_index} to {\tt nodes(node_current_index)\%objects_needed_index})
  \item and that the current node depends on the object
  \\    (add {\tt node_current_index} to {\tt objects(object_index)\%nodes_depend_index})
  \end{itemize}
}
    
\entry{[sub] object_provide(`name')}{
  Provides an object known by its {\tt `name'}
  (actually, see \link{object_provide_by_index}). 

  This uses \link{object_index_or_die}.
}

\entry{[sub] object_provide_by_index(`index')}{
  Provides an object known by its `index'.
  The objects needs either to be valid or to have a catalogued building node to be called.
  \vspace{1em}

  In the code:

  \begin{itemize}
  \item if valid, return
  \item else execute the building node
  \\    ({\tt call node_exe_by_index(objects(index)\%node_create_index)})
  \end{itemize}
}

\entry{object_provide_in_node}{}

\entry{object_provide_in_node_by_index}{}

\entry{[sub] object_modified(`name')}{
  Validate an object known by its {\tt `name'}
  and invalidate all children (known by their {\tt `index'}),
  actually, see \link{object_modified_by_index}.

  This uses \link{object_add_once_and_index}.
}

\entry{[sub] object_modified_by_index(`index')}{
  Validate an object known by its {\tt `index'}
  and invalidate all children (known by their {\tt `index'}).
  \vspace{1em}

  In the code:

  \begin{itemize}
  \item validate the object
  \\    ({\tt objects(index)\%valid = .true.})
  \item invalidate all children created by nodes depending on the object
  \\    (This uses \link{object_invalidate_by_index} 
        on all {\tt nodes(objects(index)\%nodes_depend_index(node_i))\%objects_create_index}
        for all the {\tt node_i} in {\tt objects(index)\%nodes_depend_index}.)
  \end{itemize}
}

\entry{[sub] object_alloc(`name',`object'[,dimensions])}{
  Allocate (or: reallocate) and associate (or: reassociate) an object.

  This uses \link{alloc} and \link{object_associate}.
}

\entry{object_write(`name')}{}

\entry{object_write_2(`name')}{}

\entry{object_write_by_index(`index')}{}

\entry{object_save(`name')}{}

\entry{object_restore(`name')}{}

\subsubsection{Deeper machinery}

In this section are described in a glossary manner the more core commands of the MED implementation in CHAMP.

\vspace{1em}

\entry{[fun] object_index(`name')}{
  Searches through all objects to find the {\tt `index'} corresponding to {\tt `name'}.
  Returns the index or 0 if object {\tt `name'} is not catalogued.
}

\entry{[fun] object_index_or_die(`name')}{
  Same as {\tt object_index}, but will die if the object is not catalogued
  (if {\tt object_index} return 0).
}

\entry{[sub] object_add(`name')}{
  Catalog a new object.\\

  In the code:

  \begin{itemize}
  \item add an entry to the {\tt `objects'} array,
        with {\tt `name'} as name and default values for other data.
  \end{itemize}
}

\entry{[sub] object_add_and_index(`index')}{
  Catalogs a new object that has no name and returns its (newly produced) index (actually, see \link{object_add_once_and_index}) .
}

\entry{[sub] object_add_once_and_index(`name',`index')}{
  Returns the {\tt `index'} corresponding to a {\tt `name'}.
  Assigns a new one if needed,
  i.e. catalogs the new object (actually, see \link{object_add}).
  This is used by all MED front-end routines.
}

\entry{[fun] object_valid(`name')}{
  Returns whether the object is valid or not
  (actually, see \link{object_valid_by_index}).

  This uses \link{object_add_once_and_index}.
}

\entry{[fun] object_valid_by_index(`index')}{
  Returns the value of {\tt objects(index)\%valid}
}

\entry{[sub] object_valid_or_die(`name')}{
  Dies if object not valid (uses \link{object_valid})
}

\entry{object_depend_valid_by_index}{}

\entry{[sub] object_invalidate(`name')}{
  Invalidates an object known by its {\tt `name'},
  actually: see \link{object_invalidate_by_index}.

  This uses \link{object_add_once_and_index}.
}

\entry{[sub] object_invalidate_by_index(`name')}{
  Invalidates an object known by its {\tt `index'}
  and all its children (known by their {\tt `index'}).
  \vspace{1em}

  In the code:

  \begin{itemize}
  \item if all are already invalid, return
  \item if object is freezed, return
  \item invalidate the object and its building node
  \\    (i.e. {\tt objects(object_index)\%valid = .false.}
         and {\tt nodes(objects(object_index)\%node_create_index)\%valid = .false.}
  \item invalidate all objects created by nodes depending on the object
  \\    (This uses \link{object_invalidate_by_index} itself on all
         the object indexes in {\tt nodes(objects(object_index)\%nodes_depend_index(node_i))\%objects_create_index}
         for all {\tt node_i} in {\tt objects(object_index)\%nodes_depend_index}.)
  \end{itemize}
}

\entry{[sub] object_associate(`name',`object'[,dimensions])}{
  Associate a pointer to an object
  
  This uses \link{object_add_once_and_index}.

  In the code:

  HERE
}

\entry{[sub] object_deassociate(`name')}{
  Deassociate a pointer and an object.

  This uses \link{object_add_once_and_index}.

  In the code:

  HERE
}

\entry{[sub] object_associated_or_die_by_index(`index')}{
  Dies if the object is not associated
  (i.e. if {\tt objects(index)\%associated} is {\tt false}).
}

\entry{[sub] object_alloc_by_index(`index')}{
  Allocate pointer to an object.
}

\entry{[sub] object_release(`name',`object')}{
  Deallocate and deassociate an object

  This uses \link{release} and \link{object_deassociate}.
}

\entry{[sub] node_exe(`name')}{
  Executes a node known by its {\tt `name'}
  (actually: see \link{node_exe_by_index}).

  This uses \link{node_index}.
}

\entry{[sub] node_exe_by_index(`index')}{
  Executes a node known by its {\tt `index'}.
  \vspace{1em}

  In the code:

  \begin{itemize}
  \item if node is valid, return
  \item provide all the needed objects of the node
  \\    (This uses \link{object_provide_from_node_by_index}
        on all {\tt nodes(node_index)\%objects_needed_index}.)
  \item execute the current node.
  \\    (This uses \link{exe_by_address_0}, a C routine.)
  \item validate the current node
        and all its created objects
  \\    (i.e. all objects in {\tt nodes(node_index)\%objects_create_index}
         are set to be valid).
  \end{itemize}
}

\entry{object_write_no_routine_name(`name')}{}

\entry{object_write2_no_routine_name(`name')}{}

\entry{object_restore_by_index(`index')}{}

\entry{object_freeze(`name')}{}

\entry{object_zero(`name')}{}

\entry{object_provide_from_node_by_index}{}

\entry{object_modfied2_by_index}{}
    
\subsubsection{Simple wrappers (not MED machinery)}

\entry{alloc}{alloc=allocate}

\entry{release}{release=deallocate}

\entry{alloc_range}{}

\entry{alloc_test}{}

\subsubsection{Averages}

\entry{object_average_request(`name')}{}

\entry{object_block_average_request(`name')}{}

\entry{object_variance_request(`name')}{}

\entry{object_covariance_request(`name')}{}

\entry{object_error_request(`name')}{}

\entry{object_average_define(`name',`name_average')}{}

\entry{object_average_unweighted_define(`name',`name_average')}{}

\entry{object_block_average_define(`name',`name_average')}{}

\entry{object_average_walk_define(`name',`name_average')}{}

\entry{object_variance_define(`name',`name_variance')}{}

\entry{object_covariance_define(`name1',`name2',`name_covariance')}{}

\entry{object_error_define(`name',`name_error')}{}

\subsection{About wavefunction and derivatives}

Consider

\begin{equation}
  \Phi=\sum c_I \ket{C_I}=\sum c_I \sum c_{k_I} \ket{k_I} = \sum c_I \sum c_{k_I} \ket{\uparrow_{k_I}}\ket{\downarrow_{k_I}}
\end{equation}

\noindent
and the excitation of a determinant:

\begin{equation}
  \hat{E}_{kl}\left(\ket{\uparrow_{k_I}}\ket{\downarrow_{k_I}}\right)
=\hat{E}_{kl}\left(\ket{\uparrow_{k_I}}\right)\ket{\downarrow_{k_I}}
+\ket{\uparrow_{k_I}}\hat{E}_{kl}\left(\ket{\downarrow_{k_I}}\right)
\end{equation}
.

We have the following things (all the following information can be found in {\tt orbitals_mod.f90} and {\tt deriv_orb_mod.f90}):
\\

\textbf{On the wavefunction:}

\line{{\tt csf_coef(:)}}{The CI coefficients, $c_I$}
\line{{\tt cdet_in_csf(:,$c_I$)}}{The coefficients of the determinants in the CSFs, $c_{k_I}$}
\line{{\tt iwdet_in_csf(:,$c_I$)}}{The determinants, $\ket{k_I}$}
\line{{\tt det_to_det_unq_up(ndet)}}{Map between each determinant $\ket{k_I}$ and its unique up determinant $\ket{\uparrow_{k_I}}$
\\(this has a ``down'' counterpart {\tt det_to_det_unq_dn} not mentionned here, as most of the following objects do).}
\\

\line{{\tt det_unq_orb_lab_srt_up(nup,ndetup)}}{Label (in the total orbital spectrum) of the orbitals in each unique determinant.}
\line{{\tt orb_pos_in_det_unq_up(norb,ndetup)}}{Position of the orbital in each unique determinant (0 if not in it).}
\line{{\tt orb_occ_in_det_unq_up(norb,ndetup)}}{(t/f) is the orbital occupied in each unique determinant?}
\line{{\tt orb_*_in_wf(norb)}}{(t/f) is the orbital *=occ,cls,opn,vir in the wavefunction?}
\line{{\tt orb_*_in_wf_nb}}{Number of *=occ,cls,opn,vir in the wavefunction.}
\line{{\tt orb_*_in_wf_lab(orb_*_in_wf_nb)}}{Label (in the total orbital spectrum) of each *=occ,cls,opn,vir orbital.}
\\

\textbf{On the optimization:}

Firstly, some orbitals can be excluded from the optimization space (via the input):

\line{{\tt orb_opt_nb}}{Size of the actual optimization space.}
\line{{\tt orb_opt_lab(orb_opt_nb)}}{Label (in the total orbital spectrum) of the orbitals to be optimized.}

By default, all orbitals are in the optimization space, so that by default {\tt orb_opt_nb = orb_tot_nb} and {\tt orb_opt_lab=Id}.
\\

Secondly, some excitations can be forbidden (via the input):

\line{{\tt orb_ex_forbidden(norb,norb)}}{(t/f) is the exciation involving the two orbitals forbidden?}

By default, none are.
\\

Consequently, there is the need for the following objects:

\line{{\tt orb_opt_*_nb}}{Number of *=occ,cls,opn,vir to be actually optimized.}
\line{{\tt orb_opt_*_lab(orb_opt_*_nb)}}{Label (in the total orbital spectrum) of each *=occ,cls,opn,vir orbital to be optimized (via {\tt orb_opt_lab(orb_opt_nb)}).}
\\

\textbf{On the excitation:}

From this, we can gather information on the excitation:

\line{{\tt param_orb_nb}}{Number of unique parameters (i.e. of unique devrivatives).}
\line{{\tt single_ex_nb}}{Total number of excitations (some may be redundant)}
\line{{\tt ex_orb_ind(param_orb_nb)}}{Index (in the total number of excitations spectrum) of the unique parameter.}
\line{{\tt ex_orb_ind_rev(param_orb_nb)}}{When an excitation is redundant, it is the reverse of a unique one. This is the corresponding index (in the total number of excitations spectrum) of the excitation reverse of a parameter.}
\line{{\tt ex_orb_1st_lab(single_ex_nb)}}{Label (in the total orbital spectrum) of the origin orbital involved in an excitation (via {\tt orb_opt_*_lab}).}
\line{{\tt ex_orb_2nd_lab(single_ex_nb)}}{Label (in the total orbital spectrum) of the destination orbital involved in an excitation (via {\tt orb_opt_*_lab}).}
\line{{\tt orb_mix_lab(norb)}}{(t/f) is the orbital involved in an excitation?}
\line{{\tt orb_mix_lab_nb}}{Number of orbitals involved in an excitation.}

Note that {\tt ex_orb_ind} maps the orbital parameters to the single excitations.
Indeed, for act-act with orthonormality imposed, there are subtleties: some single excitation are not free orbital parameters but reverse excitations stored in {\tt ex_orb_ind_rev}.
Otherwise, we have {\tt ex_orb_ind=Id} and {\tt ex_orb_ind_inv=0}.
\\

\textbf{On the excited determinants:}

%OLD: \line{{\tt det_ex_orb_lab_up(nup,single_ex_nb,ndetup)}}{List of orbitals in this unique determinant, once excited.}
%OLD: \line{{\tt det_ex_orb_lab_srt_up(nup,single_ex_nb,ndetup)}}{Sorted list of orbitals in this unique determinant, once excited.}
%OLD: \line{{\tt det_ex_orb_lab_srt_sgn_up(single_ex_nb,ndetup)}}{Sign of the permutation to sort this unique determinant, once excited.}
%OLD: \line{{\tt ex_up_is_zero(single_ex_nb,ndetup)}}{(t/f) does this excitation concern this unique determinant?}
%OLD: \line{{\tt is_det_ex_up(single_ex_nb,ndetup)}}{(t/f) is this unique determinant, once excited, actually an excited determinant?}
%OLD: \\

\line{{\tt iwdet_ex_up(single_ex_nb,ndetup)}}{Label of this excited determinant (either label of the ground-state unique determinant or of the excited unique determinant, depending on {\tt is_det_ex_up}).}
\line{{\tt det_ex_unq_up_nb}}{Number of unique excited determinants.}
\line{{\tt iwdet_ex_ref_up(det_ex_unq_up_nb)}}{Reference unique ground-state determinant that spwans this unique excited determinant.}
\line{{\tt det_ex_unq_orb_lab_up(nup,det_ex_unq_up_nb)}}{List of orbitals in this unique excited determinant (see {\tt det_ex_orb_lab_up}).}
\line{{\tt det_ex_unq_sgn_up(det_ex_unq_up_nb)}}{Sign of the permutation to sort this unique excited determinant (see {\tt det_ex_orb_lab_srt_sgn_up}).}
\line{{\tt det_ex_unq_up_orb_1st_pos(det_ex_unq_up_nb)}}{Position of the modified orbital in the unique determinant.}
\line{{\tt det_ex_unq_up_orb_2nd_lab(det_ex_unq_up_nb)}}{Label of the destination orbital in the unique excited determinant.}
\line{{\tt det_ex_unq_up(det_ex_unq_up_nb)}}{Value of the unique excited determinants (via SM from reference determinants).}
  \\
\line{{\tt det_ex_up(single_ex_nb, ndet)}}{Value of the excited determinants (via sign and either the value of the ground-state unique determinant or the value of the unique excited determinant, depending on {\tt is_det_ex_up}).}
  \\
\line{{\tt det_ex(single_ex_nb, ndet)}}{Value of the total excited determinants {\tt det_ex_up*detd + detu*det_ex_dn}.} 


% BIBLIOGRAPHY---------------------------------------------
%\newpage
\addcontentsline{toc}{section}{References}
\bibliographystyle{jchemphys}
\bibliography{biblio}

\end{document}


